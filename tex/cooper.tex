%------------------------------------------------------------------------------
%
%	cooper.tex Trading system part
%
%	INCLUDE FILE FOR LaTeX2e DOCUMENT
%
%	AUTHOR: Ari Potkonen /JARVENPAA/ Mon Jun 28 2022
%------------------------------------------------------------------------------
%         1         2         3         4         5         6         7
%123456789012345678901234567890123456789012345678901234567890123456789012345678
%-BEGIN OF INCLUDE FILE--------------------------------------------------------

\part{Cooperation}
\label{cooperation}
\index{cooperation}

\chapter{Cooperation, competition}
\label{cooperation_competition}
\index{cooperation competition}

\section{Global setup}
\label{global_setup}
\index{global setup}
Cooperation and competition.
World Trade Union\cite{WTO}, multilateral\cite{Multilateral} and
bilateral\cite{Bilateral} agreements concentrate politics, rules,
certifications, mostly for trade over the border, between the countries.

\section{Shipping logistics}
\label{shipping_logistics}
\index{shipping logistics}
There are long list of shipping companies caring practicalities for freight
like Maersk\cite{FreightShipping}, containers like MSC\cite{ContainerShipping}
and air cargo like DHL\cite{CargoAirlines}. Some companies are concentrated to
optimize port services, like AwakeAI\cite{AwakeAI} and some others like
Relex\cite{Relex} to whole sourcing and delivery network operations optimization
with cooperation and statistics based demand estimations.

\section{Standardization}
\label{standardization}
\index{standardization}
Standardization is community effort because producers have to change they
communication according to new standard and whole delivery chain has to be able
to read used communication. For example if barcode changes, barcode readers may
need to be updated too. For this common effort some organization is needed to
do coordination work so that everything works.

Global Standards 1 organization the GS1\cite{GS1Standards} concentrate daily
consumable products business data architecture, they have they own architecture
model as base for the published standardization documents from data and it's
usage they published for use besides they services. This data cover mostly
producer and vendor labels related data. Besides the public services they also
support B2B connections by offereing private services where only business
partners have access to data. Standards are done quite practical way supporting
business and own service sales making possible development of model and these
business standards. Standards leave implementation details for the producers,
vendors, traders, wholesale, logistic, shops and local distribution. Current
Serial Shipping Container Code (SSCC)\cite{SSCC} supports pallet, parcel or
case identification as logistic unit.

For product item identification there are Global Trade Item Number
(GTIN)\cite{GTIN}. That can be printed to one dimensional barcode or to two
dimensional barcode where is possible to take more information with, as
lot numer and case serial number. This kind two dimensional more informatic
barcodes are requested and already taken in use to fait against falsified
medicines \cite{FalsifiedMedicines}. EMVO\cite{EMVO} and FiMVO\cite{FiMVO}
organizations represent arrangment in EU and Finland.

Good example from standardization efforts implementation in practice in Finland
is Falsified Medicines
Directive\cite{Falsified_Medicines_Directive}\cite{FalsifiedMedicines}\cite{EMVO}\cite{FiMVO},
driven two dimension "bar codes" our groceries have agreed to take in use at
2027\cite{GS1DL2027}\cite{Finland2DcodesY27}. These new methods make possible
to have digital twin from product label information, and there can be lot and
serial numbers linked into information. Which make possible to have bookeeping
from deliveries and this way allow detection of duplicates and never created
product items, most likely falsified products. Now same methods are taken into
use more widely, for the groceries at latest 2027.

\section{Digitalization}
\label{digitalization}
\index{digitalization}
There are commodities digitally traded globally in big quantities, and consumer
goods delivered locally through digital services, but there are no any globally
taught, community driven frame agreement, clearing each industry area separately
and with due diligence driven integrations, merges, effective and for all
participants fear rules under global frame.

What ever tangible or intangible services or goods we deliver, it starts from
spacetime area around earth and is served and consumend in same or other
spacetime area around earth. Because we as humans can't manage global deliveries
on time wihtout some electric device driven service allowing we as humans to
manage our orders and offers. Most obvious is to use each diginative persons
own personal mobile device to this.

To get more concretic picture we could say that you might be capable to agree
with neigbour granny to go feed she's cat's and give some water for flowers.
But when you want to order certain vine or grapes from small village farm from
southern europe you perhaps need you mobile device or laptop to make order
happen and get parcel track code to youselves to be able to follow you order
from farmer to your home using the track code given. Because you may know
producer from your previous life, you may not need any fancy labels or boxes,
it's enough that you know product based to experience or product defining
documentation, which is the "asset" for farmer sales, and product just
fullfills the expectation. No need for fancy packaging, only what is needed is
unique package parcel number. Vendor can have web service pages like
\url{https://CountryCode/BusinessID/CustomerID/OrderID/ParcelID/ProductID},
perhaps locked with Customer ID, Order ID -pair customer knows due order
just made. Knowing those open possibility to view rest from tree-structure.

The GS1 (Global Standards 1) already does similar setup which is explained on
Phil Archer's video GS1 Digital Link Layer Cake\cite{GS1DigitalLink}.

\subsection{Improvement needs}
\label{improvement_needs}
\index{improvement needs}

Good example from low utilization of digitalization created business process
improvement possiblities is underutilization of Global Asset Identifiers (GAI):
Global Returnable Asset Identifier (GRAI)
\cite{GRAI}\cite{IFCO}\cite{EUROPOOL}
availabe in crates, and Global Individual Asset Identifier (GIAI)
\cite{GIAI}\cite{CajoMakeBright} usable for non-returnable cardboard case.
Note: Here Asset means actual physical item, and in further text we use only
abbreviations GIAI and GRAI to avoid mix to this document own asset defintion.

Logistic related companies, services provide GRAI\cite{GRAI} identified, with
the optional serial number globally identifiably numbered returnable assets;
cases, crates, totes, dolly's, roll cages, pallets and recycle services for
reusable ones. And there isn't any problem for cardboard box producer to add
laser to add GIAI\cite{GIAI}\cite{CajoMakeBright} to unfolded box blanks so
that cardboard boxes are also globally uniquely identifiable.

When boxes/cases are identifiable then vendors can add service which gives
status from box creation time or crate last washing time. Producer can have
service which shows what boxes/cases are holding after sent forward from farm.
Or all these services are combined under tradeoperator who can identify
accessing users and limit access according to users role on ecosystem. It's not
really understood business process and customer value creation possiblities
through using GIAI.

These uniquely numbered mediums opportunities are not really used in way and
scale they should be used especially there at the beginning of sourcing path
starting from producer. Producer can tie lot of information to sent box/case
just using he's mobile device having ecosystem application used on producer
role which allows to tie information to known GRAI/GIAI before any SSCC is
created. Product value can be added by adding digital information to cloud
services from origin of each product case from the origin place, original
product variety, harvesting time conditions and from the profession who did
the work. In some cases those are not meaningful, but along some products you
could add customer value just by adding information from product besides the
product. As far as we know unit price for product attached A4 size text data
max 4kB saving for example perishable products validity time is really small,
and you do not need high customer value growth to be profitable, just by
adding for consuming customer valid data digitally available besides the
product case, returnable or not.

If we again look groceries and the Fast Moving Consumer Goods (FMCG) from there
we can note that many fruits and vegetables are delivered using cases nubered
with serial number including globally unique Global Returnable Asset Identifier
(GRAI) and we could utilize it to hold details from cases under SSCC stored
pallet information by mapping GRAI and new two dimensional product ID's and
product label information together. Clear benefits arrive when carboard/paper
labels can be left out and detailed label information can be put to ePaper on
store whitout printing.

Extra benefits are gained from small farmers possibility to add different
products mixed to pallet layers and pallet extraction can map numbers again
to actual product information. This mixed pallet configuration is common for
small farmes selling they products on market places. They know when they
handpic cases from pallet on van backbox. Automated extration of multiproduct
pallets can be automated too, at the first for the pallets having same size
cases through pallet, but this is doable also for the pallets having different
cases on different layers. Extractor has to read next layer case identies to
get to know what it is picking next and then has to do layer picking according
to that. And for next layer same, detect what you are lifting from ID's and
with the pallet SSCC delivered information be avare from next layer picking
parameters. Repeat this until pallet is extracted and read case ID's from line
after extraction to get those case ID's too which were nonvisible at the center
of pallet and to get correct order before those are loaded over system carry
medium if in use. Then this GRAI information is mapped to system ID
representing case in the system. This make possible to increase transport
fillrate because we do not need several pallets stacked, instead we have single
pallet having mixed configuration and it's still possible to do automated
extraction for it. Finnish tomato farmers have been requesting these
capabilites \cite{WaterBalls}.

\subsection{Competition}
\label{competition}
\index{competition}
Besides these certain service orientated companies there are competitors who
try to maintain whole trading system alone like Amazon\cite{Amazon}, and bit
different business model eBay\cite{eBay}. We do not start to list whole global
competitor field here. Instead purpose is to point out what wholesalers
sourcing and delivery could learn from parcel delivery logistics and how
capabilities could be added into existing delivery logistics environments.
Main focus is kept around local players, even we can see global setup and global
trends there is no need to replay, repeat all those.

Digitalization is in phase where ordered product receiving end consumers do not
have facilities integraded in to their house. Possibility receive temperature
controlled grocery deliveries require temperature controlled space. This
significantly affects groceries home orders, and it's is more common in areas
and socities where someone is at home for example for childcare or doing remote
work. This limits customers and pure digital delivery operators dark stores
success in this kind areas, good example is Oda\cite{Oda} tryout in Finland.
Currently existing wholesalers shop chains try to figure out best digital home
delivery method in they existing setup. Clearly there are hand picking from
store, darkstore \& handpicking combination and automated darkstore \&
handpicking combinations. Then deliveries to picking closures or stright to
consumption point, usually private home. Courier services like Wolt\cite{Wolt}
and Foodora\cite{Foodora} offer possibilities to deliver groceries from retail
to customers nearby.

\subsection{Software trade}
\label{software_trade}
\index{software trade}

Standars can be built bit more opensourced way where cryptographic identies,
electric signatures and blockchain model offer some trust and you aren't so
much dependent from service provider... To be written (TBW).

%-END OF INCLUDE FILE----------------------------------------------------------

