%------------------------------------------------------------------------------
%
%	trade.tex Trading system part
%
%	INCLUDE FILE FOR LaTeX2e DOCUMENT
%
%	AUTHOR: Ari Potkonen /JARVENPAA/ Mon Jun 28 2022
%------------------------------------------------------------------------------
%         1         2         3         4         5         6         7
%123456789012345678901234567890123456789012345678901234567890123456789012345678
%-BEGIN OF INCLUDE FILE--------------------------------------------------------

\part{Tradesystem}
\label{tradesystem}
\index{tradesystem}
%\chapter{Digital trading, e-commerce as frame agreement backed ecosystem service platform}
%\chapter{Digital trading as frame agreement backed ecosystem service platform}
%\chapter{Frame agreement backed ecosystem service platform}
%\chapter{Frame agreement backed ecosystem}
%\chapter{Ecosystem frame agreement}
%\chapter{Ecosystem frame}
\chapter{Frame, framing}
\label{frame_framing}
\index{frame framing}

Digital trading, e-commerce as frame agreement backed ecosystem service platform.
Few ideas from digital trading, e-com\-mer\-ce platforms which traditionally has been thought too narrow mind\-ed way, concentrating to single customer usability questions, vendor viewpoint or service provider viewpoint and then drifted to existing setup. It has been enough because there is no serious competition from government or society driven frameworks. Actually this area is much, much broader what has to be covered for successful platform forming ecosystem services.

To be able to form ecosystem services for full delivery chain, from initial raw material producer to final consumer and recycling, you have to have strong value based idea from fair and sustainable way created ecosystem. Ecosystem which can be accepted globally by original producer, final customer, and society from socioeconomic perspectives. Ecosystem which is enough aligned with legal framework and government, like with EU digital strategy, so it can be accepted to be used and supported by government, and which is financed by offering clear role for financing institutions, therefore having motivation to back up technology. Then banks have clear future role. When public setup is widely copied it can't be wiped away, instead current major players are encouraged to join the movement.

To get to widely accepted and popular ecosystem services into place there has to be widely accepted rules how ecosystem services operate, how they improve every participants life. Therefore publicity especially for ecosystem own rules has to be clear, documents freely available and understandable for regular peoples. For legal framework there can be an another freely available public set of legal documents setting these clear explanations into practice.

Now there is no other needed level agreements than some "free trade agreements" -- and are those on needed level for acceptable ecosystem services? Those really are not adequate at all, or limited to too narrow niche.

Biggest risk here is that EU government really doesn't understand what they are doing. Legal body has demanding and denial legal rules mostly in mind when they do something. These both are mostly limiting and distracting ways to build anything new. They also have possibility to finance some new initiatives if they understand what they should do. Good example from EU activities is this new Cyber Resilience Act (CRA)\cite{EU_CRA}. EU should actually finance open hardware and open software reference implementations tested against major Open Source Hardware (OSH) and Open Source Software (OSS) releases and put to public repository usable with Fedora, Debian, Rasbian,.. etc. for semiautomatic and automatic deployments directly from repositories to device having hardware implementation\cite{ISO_IEC_19790} done. Understanding, financing and execution of this kind of operations is not EU strength area. For example eIDAS (electronic IDentification, Authentication and trust Services) node implementation statistics \cite{eIDASnode} show that services are implemented, but countries have they own implementations still in use, like BeID, EstEID, FINeID, so even EU has some reference implementation available countries utilize they own and cooperation seem to be hard to get working. Belgium has had quite long time this open client implementation which is brought to be part of major OSS distributions, so even this kind of open implementations exists and freely available, it seems that EU is not capable to add some localization support and take good solution into wider use.

How about then this EUDI (EU Digital Identity) wallet? 

%\section{Platform strategy creation viewpoints}
%\section{Platform strategy creation}
\section{Platform strategy}
\label{platform_strategy}
\index{platform strategy}

Platform strategy creation viewpoints.
When we look ex\-ist\-ing digital trading platforms having some kind of ecosystem around, we notice that the terms of ecosystem for single vendor are proprietary and prohibitive, many ways limiting vendor possibilities to operate on market. There are few of those platforms available and clear sense says that there is need at least for EU wide ecosystem rules as public frame agreement created in way that it could be extended globally after a while. Figure \ref{fig:TermSets} on page \pageref{fig:TermSets}, opens situation and needed frame.

\begin{figure} %[H] %\usepackage{float}
 \begin{center}
  \epsfig{figure=figures/termsets.eps,width=\textwidth}
  \caption{Existing digital trade rule sets and need for EU wide public nondiscriminating rule set for platform based ecosystem}
  \label{fig:TermSets}
 \end{center}
\end{figure}

To approach this frame agreement controlled digital trade platform idea from digital marketing strategy perspective to really be able to go details we need some clarity for thinking. There is good model for strategic analysis and planning views of digital marketing from single vendor perspective at (Heikki Karjaluoto et al. book \cite{Karjaluoto2022}, Figure 1: Book structure, Page 10). Inspired to extend this Vendor's Strategic analysis and planning of digital marketing -- cake model to cover Digital Trade Platform Marketing Strategy -layer based to this idea of foundation or cooperation managed frame agreement based platform setup having related technology selection, partnering, public relations and customer acquisition in own hand. On extended model kept that single vendor as customer, added platform layer underneath model and we got extended platform marketing cake model we can iteratively use to evaluate different players viewpoints to platform digital marketing strategy, figure \ref{fig:TradeMarketValueGrowth} on page \pageref{fig:TradeMarketValueGrowth}.

\begin{figure} %[H] %\usepackage{float}
 \begin{center}
  \epsfig{figure=figures/dTMVGstr.eps,width=\textwidth}
  \caption{Digital Trade Market Value Growth (TMVG) strategy analysis and planning}
  \label{fig:TradeMarketValueGrowth}
 \end{center}
\end{figure}

Original model round MVG (Market-Value-Growth) cake model form changed to triangles to speed up work. While adding and taking new bottom Trade platform layer as major viewpoint for this TMVG (Trade-Market-Value-Growth) cake model, where original vendor MVG model rotated half a round at the center of extended model. Side view from layers on figure \ref{fig:TradeMarketValueGrowthLayers} on page \pageref{fig:TradeMarketValueGrowthLayers}.
\begin{figure} %[H] %\usepackage{float}
 \begin{center}
  %\epsfig{figure=figures/dTMVGcut.eps,width=4cm}
  \epsfig{figure=figures/dTMVGcw.eps,width=\textwidth}
  \caption{Digital TMVG strategy analysis and planning model layers front view}
  \label{fig:TradeMarketValueGrowthLayers}
 \end{center}
\end{figure}
While this cake model is now extended for Trade platform we have to extend it's operational part mRACE \cite{SDM2022} (measured Reach Act Convert Engage) figure meant for Operational implementation of digital marketing as well (Heikki Karjaluoto et al. book \cite{Karjaluoto2022}, Figure 1: Book structure, Page 11) what is already extended model from Chaffrey RACE \cite{ChaffreyRACE} (Reach Act Convert Engage).

We have to sell our frame agreement defined way to do business for participants. It should be easier than pure technology implementation project sale because frame agreement can be defined long lasting, instead technology which changes after while, so everyone understands that investment is for longer period even technologies and process underneath could change. This mRACE model is extended with Involvement and Incorporate steps which are used on all levels for all players namely for customer, influencer and vendor. Created frame agreement and initial technical definition has to have enough benefits to roll up some business areas fully over platform services. This is platform selling for vendor. For vendor inclusion means that they invest they time and money to collaborate on frame agreement, business process and technical implementation level. Incorporation means that they select platform for they business and support platform operation and development with they turnover tied fees and perhaps be members of foundation or cooperative maintaining the platform. All these can be measured by using public code and documentation repositories having review and reporting capabilities for registered users and participating patrons representing they organization. This mRACEII (measured Reach Act Convert Engage Involve Incorporate) digital marketing operational implementation figure \ref{fig:mRACEII} on page \pageref{fig:mRACEII}. Customer and influencer cases are explained later on this document.

\begin{figure} %[H] %\usepackage{float}
 \begin{center}
  \epsfig{figure=figures/mRACEII.eps,width=\textwidth}
  \caption{Digital trade operative marketing activities on mRACEII planning model}
  \label{fig:mRACEII}
 \end{center}
\end{figure}

%\section{Ecosystem frame agreement embedded strategy creation}
%\section{Embedded strategy creation}
\section{Strategy creation}
\label{strategy_creation}
\index{strategy creation}
Ecosystem frame agreement embedded strategy creation.
As stated earlier most needed and valued participants on ecosystem are initial product or service creators and final customers who pay the bill. Everything else between there is for just business, and will participate anyway after these first two major groups are done decision to move on this platform. At the beginning we have to think big to get some idea from scalable future proof framework what is then implemented in legal agreements, clear text explanations and actual technical design, all step by step trying to implement most needed parts first, still keeping in mind wide perspective to maintain future proof implementation. It makes sense to listen original creator and final customer rights balance driving organizations like EDRi, NCUC, NPOC to get as high acceptance for platform and frame agreement as possible. This requires different viewpoints opinions regular iteration and improvements implementation to frame agreement and actual platform services as seen on figure \ref{fig:iteration} on page \pageref{fig:iteration}. Similar iterative strategy process introduced at NRWA Conference by N.N. \cite{NNlostAtNRWC2022}, and there are also older documents explaining strategy loops.

\begin{figure} %[H] %\usepackage{float}
 \begin{center}
  \epsfig{figure=figures/iterat.eps,width=\textwidth}
  \caption{Iterative process}
  \label{fig:iteration}
 \end{center}
\end{figure}

These iterative loops has to be repeated regularly based to followup need and need to be able to adjust strategy. frame agreement and platform implementation according to changes found from business environment; legal environment, technology sector, market area, social environment, trends, standardizations, and logistics services operating on market area.

Bottom layer holding platform trade technology services, and it's the most investment intensive layer because it's implementing and operating technical framework for all upper layer defined needs and services. This includes terminal software for producer, vendor, archival, logistics, toll, tax, vendor, trade, banking, sourcing, maintenance, recycling services and service integrations.

Top of bottom layer sits market area, which holds all regions, countries business areas and customer groups platform supports are enabled for. Each region, technology area and customer group may have they own requirements, which you have to take in account when creation platform strategy. It may also include decision that certain business area or regions are left without support from platform side. And it may be wise to limit supported trading to traditionally widely accepted business modes without cheating and room for scalpers to operate.

Then comes vendor needs and services layer where are integrations to external system's like social media and vendor own systems.

At the top there are vendor own digimarketing strategy, which should be doable to do with the lower layers support and cooperations with vendor and platform support personnel. To support this cooperation work we have added additional query questioning layer to operative marketing, figure \ref{fig:mRACEIIQ} on page \pageref{fig:mRACEIIQ}, to collect data needed to support strategy iteration loops, figure \ref{fig:iteration} on page \pageref{fig:iteration}.


\begin{figure} %[H] %\usepackage{float}
 \begin{center}
  \epsfig{figure=figures/mRACEIIQ.eps,width=\textwidth}
  \caption{Digital tradesystem operative marketing activities on mRACEIIQ planning model}
  \label{fig:mRACEIIQ}
 \end{center}
\end{figure}

%\begin{figure} %[H] %\usepackage{float}
% \begin{center}
%  \epsfig{figure=figures/mRACEIII.eps,width=\textwidth}
%  \caption{Digital tradesystem operative marketing activities on mRACEIII planning model}
%  \label{fig:mRACEIII}
% \end{center}
%\end{figure}

Figure \ref{fig:layering} on page \pageref{fig:layering} shows global, even astronomic scale trade area information layering, mixing things purposefully in figure \ref{fig:layering} on page \pageref{fig:layering}, to create illustration from frame agreement controlled information space where agreed roles and agreed trading manner based transactions are driven over globally, or even astronomically, scalable technology services created for purpose. And here all official product defining documents; fulfillment statements, installation, maintenance, user and quick manuals, guarantee and service contact information are only on electric form, and customer should be able to do purchase decision based only to those documents information. This product and related documentation defining information we can call as ASSET, seen in figure \ref{fig:layering} on page \pageref{fig:layering}. This heavy documentation called asset describes product items and relations between multipart product items, it usually is not actual product, but more a market definition from product, availability areas and dependencies to other area availability.

\begin{figure} %[H] %\usepackage{float}
 \begin{center}
  \epsfig{figure=figures/layering.eps,width=\textwidth}
  \caption{Tradesystem spacetime layering}
  \label{fig:layering}
 \end{center}
\end{figure}

Below the visible layer vendors can have they own for customer invisible layer where they prepare they product availability on visible market space (S. Renken at al. \cite{RenkenNRWC2022}). When product is available vendor gives CERTIFICATE for trade. Certificate is promise from product availability to deliver in certain area, place and time frame as vendor see to provide asset defined product to market using frame agreement agreed trade method, product definitions on asset and tradeable certificate from product or service availability on market. OSS licenses also form certificate from unlimited availability under certain terms without payment, having zero price.

If we have to illustrate trade transaction in tradesystem spacetime it's like lightning stroke through global, even celestial space, when orbiters like airplanes, space stations, moon, mars, etc. have economic online transaction activities. From customer needs and money resources system does resolution to consumption place or storage space on existing infrastructure. If resolution finds non-broken chain of dependencies fulfilling the need in consumption, certificates from all needed components with correct availability are existing, and on time cumulated price is not exceeding available monetary resources then purchase, with from included certificate information known availability times and places is offered for customer to purchase as transaction approval decision.

\begin{figure} %[H] %\usepackage{float}
 \begin{center}
  \epsfig{figure=figures/resolut.eps,width=\textwidth}
  \caption{Resolution from assets and certificates in the spacetime of tradesystem}
  \label{fig:resolut}
 \end{center}
\end{figure}

How to deal with this scale setup that it doesn't collapse from the beginning? From agreement point of view is good to start from frame agreement and some business area subcontracts holding details for area. From technological perspective there are some technological basis which scales well and makes possible cumulate huge masses of information without fear to mixing it. Certainly we have to offer dedicated views for customer, but those can be just limitations on browsing asset covering all offers. We should not do silos until it is necessary, there is enough or those already.

\begin{figure} %[H] %\usepackage{float}
 \begin{center}
  \epsfig{figure=figures/terms.eps,width=\textwidth}
  \caption{Frame agreement terms for fluent frictionless trade}
  \label{fig:terms}
 \end{center}
\end{figure}

Idea is to define fundamentally global service applicable asset structure, which is used to define offer. To get offer compelling easy to administrate and operate we has to have frame agreement for vendors participating to market. Besides offer we introduce certificates which are vendor given guarantees that they are ready deliver what they selected to be shown in asset. Then certificate is actually thing which is traded. Different sectors like local hospitality, or service stations etc. can be represented as limited views from whole asset. There are lot of artificial intelligence developed to be able to do easy limitations to asset visibility for most needed and relevant ones at time. You could compare to google maps service, which you can quite easily to use to search cafe's etc. from local area.

%\section{Single minimum asset}
\section{Minimum asset}
\label{minimum_asset}
\index{minimum asset}
Single minimum asset.
Asset in generally is something where customer user want to use hes own resources time, space and money. Asset in business is something what vendor can offer which has value and can be changed to money. Single minimum asset is seen as smallest configuration of items like "Minimum Viable Product" on vendor offer customer has seen to be ready to commit use hes own resources, like money, time and space. Asset configuration may hold one or more product items or services to reach for customer valuable asset configuration which is what is normally added to vendors offer. It may also hold free zero priced items customer is willing to consume using hes own resources. While we are defining fully digital asset, for physical products and services asset includes marketing material and technical documents defining actual physical or service item included in asset. Under the ecosystem services asset configuration need to be linked to all official documents product must have on that market. Besides mandatory documents there can be optional material, and linking from other mediums like social media to asset. Combined assets are forming vendors offer on market.

\begin{figure} %[H] %\usepackage{float}
 \begin{center}
  %\epsfig{figure=figures/asset.eps,width=4cm}
  \epsfig{figure=figures/asset.eps,width=\textwidth}
  \caption{Asset is item fulfilling consumer need requirements}
  \label{fig:asset}
 \end{center}
\end{figure}


%\section{Certificate as means of exchange}
\section{Certificate}
\label{certificate}
\index{certificate}
Certificate as means of exchange.
While creating fully digital ecosystem services there is need to separate product definition "asset" and product existence and availability promise "certificate", which is vendor promise from product availability against customer own resources money, time and space.

%\section{Where the products are ?}
\section{Product}
\label{product}
\index{product}
Where the products are ?
Product is defined in asset and each product item is then available as certificate states. Availability place can be defined in both asset and certificate. Asset item describes general availability on market as product is meant to be offered on regions. Certificate refines particular product item availability details, license, business model, pricing mechanism and transferability or resellability.


%\section{Asset as dependency tree}
\section{Asset dependency}
\label{dependency}
\index{dependency}
Asset as dependency tree.
One product, service etc. in commerce is forming tree which branches form continuous non-broken chains from sources and requisites to combined offer, merchandise or service, called asset item which is one part of whole asset what is offered as vendor offer on market for market. On market asset trees form forest where trees could be build over others and then being dependent from others.

From vendor side this chain could include several farmers and logistic operators, or subcontractors, parts, bill of material lists, their assembly service and assembled product delivery. Products from assets could form bundles. Bundles which are treated as tree structure capable to forming new product asset item in asset item tree, and which is depending from availability of it's leaf component items, actual items in bundle. Asset subtree can include physical products, software, service, farmers environmental certificates, industry player sustainability certificates etc. For locally created products there could be dependencies to taxation nodes, language packages and toll nodes which needed to applied when exported/imported to certain area or region.

\subsection{What is really the hardest part of creating good asset for market?}
\label{selfishness_obstacle}
\index{selfishness obstacle}
Each vendor have they own ideas from they business, whom are they companions, competitors and in what terms they product is available and for whom. This holds even they ideas are stupid, hindering whole market growth and they own profitability too. They will keep they rules even it makes impossible to form compelling electric commerce asset for market. Who cares, "have been living like this for years" until today. So short term advantage and fear for competition is hindering some companies, until they are finding themselves be outsiders from market, really loosing they business.

\subsection{How we know when we have a good asset?}
\label{asset_goodness}
\index{asset goodness}
Resolution from customer needs to set of asset configuration candidates fulfilling customer needs should be like thunderstorm lightning strike, fast and comprehensive, all meaningful branches are gone trough and we have zero on more resulting branches from customer need requirements filling asset item findings as solutions for customer needs.

\subsection{How to limit needed customer selections?}
\label{asset_search}
\index{asset search}
Asset structure has to support dependencies to standards, like product sustainability classification, production geolocation, what customer can use limit resolution target besides the normal; price, brand, place, size, availability. Customer should be able to define filter stack per product category. Or few filter stacks he could select when doing selections for certain purposes. Natural Language Processing (NLP) and Artificial Intelligence (AI) can be used to maintain these filter stacks besides manual creation and management.

\subsection{How we can get this kind of clean asset?}
\label{asset_cleaness}
\index{asset cleaness}
Because of human opportunistic tendencies\cite{SevenSins} we have to have strict and straight rules defining how different vendor assets can be combined to be offered for customer. To form this kind of rules we have to have agreements.

%\section{Frame agreement}
\section{Agreement}
\label{frame_agreement}
\index{frame agreement}
Frame agreement.
Ecosystem based to Electric Market Frame Agreement (EMFA) and independent legal entity to maintain it; limited company, foundation or cooperation! This is important to do now when forming new digital trading based market. It hast to be created at the start when vendors are applying to market.

\subsection{Why to setup frame agreement?}
\label{frame_agreement_why}
\index{frame agreement why}
There are lot of software and media products etc. with owning business own  opinions which may include idea to negotiate terms customer by customer. We really do not want to negotiate that for this you product; Can we sell, rent, combine it to other vendor products and which vendors, if any. IT IS WASTE OF TIME AND BUSINESS to discuss with these NO Sayers. There is lot of consumers who have stopped to use paid content, because services offer only some part from need, and to get wanted offer you have to purchase and utilize several services. Then they purchase none, or wait years price to go bargain to the bottom before purchase if even then. Therefore we setup an frame agreement. You agree that if you get money against your product then customer, or other vendor, even competitor can utilize you offer as whole or even only one component from your product as significant part of they own convenience or as part of created product offer if they wish to do so and get value from it. Major requirement from frame agreement is that "negotiation" is done only once against the platform.

Platform support for imported business models, supply and delivery chain details, can be extended in good cooperation to get most optimal process setup at operation point, but major requirement is actually "accept frame agreement or go somewhere else to sell your products" - Don't compromize platform promise from unified market even cooperation is big part of friction removal mentioned in Ojanper\"a Platform Strategy book\cite{ojanpera2021platform}, it should not allowed to steer platfrom to ending sidetrack. From platform operator perspective it's curical to select included business areas in right order at right time so that generic income financing is mostly enough for the platform own installation growth. It's good to cooperate and leave new included economy sectors own friction removing and business enabling infrastructure investments for vendors interested to do they major business at that sector. From platform perspective it's enough to agree on frame level, create needed Application Programming Interfaces (API's), protocols to support integrated business process on platform level.

Certain business area support has to be done once and well for all vendors to keep cost down and business profitable. Can't and won't integrate every vendor own supply, delivery chain integrations into plaform. Vendors can use they existing automations for they offer creation but offer on platform is tied to frame agreement. Asset structure itself has inbuilt possibility to create new combined assets based offer. Purchace triggers components delivery. For business scale and profitability platform will offer common possibilities for business automation under frame agreement. Frame agreement has define and support these combined offers, new business creation and existence in long run.

Secondary targets on frame has reduction of artificial value chains created by discriminating certain customer classes. On highly educated areas is common trend that even customers are very well capable and eligible to do professional things the local stores and store chains do not sell products, tools and supplies at all for individual consumers. Same is for professional information, it's hided under marketing trash pushed through search engines and social media. Therefore platform is needed where detailed official product information is stored over very effective, economical and scalable way in asset structure, which allow more service oriented companies to automate sales, do lot of management, sales batch splitting effectively by themselves or by using service provider from market and overcome this trend where individual customers assumed smaller purchases is assumed to have bigger information and service need is discriminated by default. Frame agreement is needed to get rid of customer class discrimination. If customer is capable to find vendor or service chain creating product or service on price customer is willing to pay then sale transaction should also happen.

Minor target is to empower regional players to proof and brand they own products for global market by bringing classification services to market so that vendor can improve they own asset information for other region by using regional classification services and then social marketing called automarketing to bring products on peoples awareness. Currently delivery chains from region to other include rebranding copy products and this rebranding may include classification service or just distributing vendor statement from fit to purpose. Anyhow rebranding copy product usually creates some unnecessary overhead which can be overcome by offering good visibility to standards and classification services to prove they products quality from start and avoid proofing several times from several distributors.

\subsection{What to agree with the Frame Agreement?}
\label{frame_agreement_what}
\index{frame agreement what}
Frame agreement purpose is to make safe non-discriminatory market having all normal standardized business models (sale - purchase - resale, maintain - serve - subscript - transfer, lease - rent, share, ..) and related pricing methods (cost-plus, time based, variable, dynamic) supported in way anyone in market can create combined product, collect ingredients, add own effort, use it as such or sell as new product. Everything can be mixed directly from market and sold further with configuration created vendors liability as long as configuration component bills are paid.

Agreement define certificate what vendor brings to market. Defined certificate is the sold thing in market. Certificate against the goods and services are delivered. Frame agreement also defines technical rules and methods implemented as platform library methods what used when digitally protected medias are used. EU Governmental Digital Preservation takes unprotected originals to protected archive and create protected version to trade operators public archive. Vendor has to deliver clear originals with documentation and tested scripts for protected version creation to Digital preservation\cite{EULegalDepositScheme}, which then runs scripts to create protected version from originals to market. Originals are stamped to keep protected copyright time, and then released automatically to public space. Sold certificate is mortgaged at bank value account book entry\cite{BookEntry}\cite{LEX_2017/384}\cite{EU_Settlement} to deliver media or software keys to device to unlock software or media on they device.

Frame agreement include sections for the governmental players e.g. toll, taxation, tax is one automatically managed dependency in asset.

Agreement also has to have clear rules from mechanisms used to protect from speculative players, in practice meaning limitation of bulk purchase for goods having much other customers too. Basically vendor could set bulk purchase limits suitable to they own product and production volatility tolerance, to maintain they product brands availability to wide audience without cream peelers caused shortage harming product brand and future profitability. Trade operator has to have automatic dynamic default bulk purchase limitations, like offering vendor widely distributed certificate serie or purchase redirection to vendor human manged bulk sales team and automatic follow up so that badly behaving speculative players purchases are automatically limited, banned for period or even kicked out from market. Market is meant for fluent ecosystem creation not for speculator handy tools. For example football match tickets resale could be limited to average family size for customer or legal instance like limited company. So if company purchases more than 10 tickets it's not able to sell further those without separate agreement, only possibility is to return those back with original price on time limitations when they still are applicable to return and refund. 

Frame agreement has also include mechanism to limit market damage when software or media or other company having delivered stock and certificates on market go bankruptcy or liquidation by hostile market operation. Sales are automatically continued to protect other vendors business build over these products. Tangible product sales stop when certificates are sold, intangible product certificates are generated automatically for sales as before bankruptcy or hostile takeover. Money from sales is delivered to estate or usurper. In short vendor can update software product on market and demand customers to use latest version, but it can not remove product or earlier versions from market others are already based they business. Hostile upgrade operation is also neglected by allowing older version use. Only product security reasons can be used to reason product ban from market, still others are allowed to use component under they own products if they can overcome security issues in they own configuration. For example by using snap packages for software to allow cherry pick used libraries and configurations more independently from host platfrom update cycle, and this way allows to prolong software component use, get time for business to adabt and migitate possible financial risks from sudden change in ecosystem asset.

\section{Roles and terms}
\label{roles_and_terms}
\index{roles and terms}
To be able to define how ecosystem really works we have to define roles and terms we can define. While definition es clear for reader he could understand the whole story where these roles and terms are used.

\subsection{Customer}
\label{customer}
\index{customer}
Customer is actor willing to get something from offered asset items and might be willing to pay something to get what he is willing to get. It might be that offered assets hasn't have needed and wanted items configuration available. One simple example is customer needing one or two items from products sold only in boxes holding way too many peaces of items to be sensible for customer to buy.

\subsection{Recycle fee}
\label{recycle_fee}
\index{recycle fee}
Most customers do not have persistence to care used goods to recycling without any incentive after they own use is ended. Per product defined recycle fee guarantees that someone is interested to care from recycling if customer doesn't and used product value is negative without attached recycle fee. Recycle fee purpose is to make used product value so positive that recycle happens.

\subsection{Recycle services}
\label{recycle_services}
\index{recycle services}
Recycle services care professionally from recycling. Actual recycling depends on product and customer value left in it. Recycling services classify, perhaps maintain and resale product. If original use continue then recycle fee is still attached to product. When product original use ends then recycle fee is used to make possible to proceed negative value product further processing on recycle services so that there are no unnecessary environmental load left after product original use ends. Recycle services can have trade operator role for reusable products, or they can use original trade operator to do the used product resale under used category.

\subsection{Influencer}
\label{influencer}
\index{influencer}
Influencer is people doing commercial or semi commercial marketing, usually producing marketing material to social media environment. When this marketing material is not directly vendor created, creator is not directly legally responsible from any errors material may have. Still usually these social groups trust more to these reference group materials they feel to be as customer reviews for products, even it's not whole truth. Ecosystem should offer clear reference mechanism to reference products, product defining official vendor documents in asset from social media direction.

\subsection{Vendor}
\label{vendor}
\index{vendor}
Vendor is legal entity agreed ecosystem frame agreement, taking product responsibility and gets right to added asset items to offered product configurations sold against certificates. Asset forms the content defining the sold configuration content. Certificate defines the availability and trade system gives price for certificate depending the pricing modes vendor has selected for configuration items.

\subsection{Bundle}
\label{bundle}
\index{bundle}
Bundle is from vendor offered sold items created configuration which items legal responsibility stays on each vendor selling the item. Bundle is meant for several uses. Vendor can utilize it for marketing campaign he references from campaign. It can be used as customer purchase list when customer creates configuration he want to buy. It can be used as influencer-customer created product combination he/she is referring from social media while he communicates for hes reference group or followers.

\subsection{Bank}
\label{bank}
\index{bank}
Banks has to be extended from value-account perspective. Bank works as trade-operator doing trade transactions purchasing and selling certificates from trade-operator, bookkeeping is kept on value account. Groceries are kept on own folder, product having guarantee on own folder and resellable software licenses etc. on it's own.

Besides the software etc. vendors license and locking management bank do certificate mortgage against for the device generated keys.

Teleoperator can also play role of bank, and then should be legally treated as bank or relevant financial institution from local legal law perspective. In many developing countries teleoperators do have banking services.

\subsection{Trade operator}
\label{trade_operator}
\index{trade operator}
Besides bank there has to setup for trade operator services. Trade operator takes in asset trees and linked materials which all are checked and archived at the beginning. Trading starts when operator lifts archived material visible to cloud edge. Customer can browse and do resolution, resolution for wanted good for certain target like area, device, delivery day, etc. other dependency needed to be met. 

Actual trading can be made when vendors deliver certificates, which state asset defined product availability on market. Certificates are sold for value account and usually changed immediately for goods and services. Logistic operators can also define they asset and then sell certificates for delivery slots they have.

This makes possible to create and maintain local delivery chains for market on market. Trade operator is also market supervisor, makes sure that frame agreement and it's rules are understood and accepted before entering market. It's also remove and sanction players violating generally accepted market rules.

How to make sure that someone doesn't steal or copy certificate and claim products against it? Actual product delivery is done based to proven transactions, meaning that bank verifies who has certificate on hes value account and deliver recipe including payer and delivery information to vendor.

\subsection{Trade technology provider}
\label{tehnology_provider}
\index{technology provider}

Trade technology provider is key player while it provides technology services, consulting and cooperation for trade operators, mobile terminal vendors, banks, authorities, ecosystem vendors and finally for ecosystem customers. Vendors and customers trust for fair, non-discriminating, undisturbed, future proof, ecosystem is mandatory to achieve. Therefore fully open "copy left" software components has to be used. There is still enough money coming in because banks want to stay relevant also in future, and are willing to finance non-profit organization like foundation or cooperation to do the work. Governments may also support political economy fluency and profitability improving technology creation and maintenance.

There are similar shared technologies already existing on human identification and healthcare areas where common technologies are developed and used on several vendors and service providers services. Digital- and cryptocurrencies are cutting corners, mobile terminal vendors enable mobile paying with terminals, network operators may offer banking services, retail store chains are supporting mobile payment and extra services, like electric receipts, bookkeeping from purchased certificates, especially from those still including some value like licenses and product guarantees under user account. When employer start to pay employee wage with mobile digital money then banks start to be irrelevant for some customers.

Banks are really left behind and it start to be questionable why they exists, are they relevant to customer anymore. 

Banks has to think how to stay relevant.

Therefore there will high competition from the technology pro\-viders place, from winning technology stack, ownership and monetizing it with the governments support. It is too big money making machine to let some profit hungry corporation to have monopolistic applications stack as only available option. We have to have non-profit cooperative and open setup for this.

\subsection{National Digital Preservation}
\label{digital_preservation}
\index{digital preservation}
National digital preservation legal deposit\cite{Vapaakappaleoikeus} takes unencrypted originals, sources, components and store those to protected national archive. Because author death plus $n$ years where $n$ about 50-70 is hard to follow we can take fixed time as Copyright time which is nearly 109 years. After the end of copyright period unencrypted originals, sources, components are published besides the protected ones on market asset archive. Comparing to Walt Disneys Mickey Mouse: Protection time is about 1966-1928+70 years = 108 years which is about 0xCCCCCCCC seconds being easy to remember "CCCCCCCCopyright" seconds time. Should be enough, no need to have own value for each country (Wikipedia\cite{CopyrightLengths}).

\subsection{Archival services}
\label{archival_services}
\index{archival services}
At the begin of asset creation it will be archived on archival tier services because there are no more users for item than it's creator. Asset can be on protected services as under ecosystem services.

Protected asset could work as Bill Of Material (BOM) for company work products based to they own internal needs. Protected asset could refer, depend from other products as raw materials ordered for product creation. Important thing for the vendor is possibility to get raw materials availability and delivery times for the needed components. Vendors can also cooperate and  do automation under the market hood.

Public asset is then created for the products meant to be sold for customers, and certificates can be published based from protected asset dependency resolution given delivery times and own work and own delivery time estimations at the time. It is also possible to operate with fully public internal asset.

Of course it gives some information for customer and competitors, but it might be first phase when some influencer marketing person person creates bundle definition from assets he thinks fit together and marketing it for social reference group he belongs to. Commercial asset setup is still legally on official vendors responsibility.



\section{Technologies to be created}
\label{technologies}
\index{technologies}

\subsection{Asset structure}
\label{asset_structure}
\index{asset structure}
Most important is the asset structure which allow vendors to decide which part is they internal bill of material, and from where starts they sellable product or do they offer everything as product and spare parts.

Well proven working models to look ideas for asset structure are Debian "dep" package repositories, RedHat Package Manager "rpm" repositories and Git version control, from where to do generalization to platform independent level, like using RFC 2119\cite{rfc2119} keywords, additional keywords like "FULFILLS", "FOLLOWS", "OFFERS", "PROVIDES" and what already found from mentioned rpm and dep repository structures.

There is need for asset as repository generalization as repository holding possibility to run it as block chain where all is archived and then latest are brought to cloud edge services and multiplied on service load need bases. And still structure has to hold all information to show in human readable form as well. Cryptography based block chain is best for the asset, because you can save years and years over the old asset without problems, just move less accessed parts to archive servers.

For future needs all conformance, guarantee, manual and quick setup papers has to be part of asset even fruitlabels with possible variations are stored in electrical from as part of asset. Then sold certificate tied to delivered lot refers to label document defined in asset. If new quality is introduced then asset is extended, top nodes recreated and written to asset repository.

Due to nature of asset as block chain stock, it is easy to offer "barefoot-network" -delivery for selected media bulk on rural areas where wireless network coverage and available bandwidth do not allow to transfer bulk over the air and wired connections do not exists. There usually electricity availability is also scarce resource and peoples go to charging stations or other places where electricity is available. Then selected parts of asset tree can be carried with the terminal and shared through USB, bluetooth, wlan while charged. Transfers can be two directional uploading some assets to charging station and downloading something else. For example software updates, e-books, music and movies can be transferred. Maybe existing charging stations can be upgraded with mobile managed physical mail box closures too then there is possibility two direction tangible products delivery too withing application.

\subsection{Resolution process}
\label{resolution}
\index{resolution}

Resolution process does customer need based selection from assets. First it does asset view limitation to human understandable set based to predefined selections like non-lactose, ecologically cultivated food. These limitations can be searched as certifications or standards on asset the selected asset is stated to fulfill.  

If we think customer device software asset resolution, then resolver checks which hardware and software it's running on and then do the resolution from wanted configuration to platform underneath. If non-broken dependency chain is found and there are certificates available to purchase for components needing those, then resolution is done successfully and purchase, download and installation process can be done as final step of resolution.  

\subsection{Client application}
\label{client_application}
\index{client application}
\index{application, client}

Client application will be capable to do resolutions based to categories, classification, dependencies, price range. Most important capability is capability to follow dependencies from the wanted to platform, interface, standard, list, etc. Most simplest is to look food name, and got list of vendor made ready products and recipe with the dependent ingredients. One product or whole list can be purchased. With the rules you could limit what will be shown: cheapest, brand,.. most used.. available now. Artificial Intelligence (AI) and Natural Language Processing (NLP) can be used to collect reduction related rules. Client application can also utilize into terminal pre-loaded assets or from peer transferred assets without network coverage. For the protected content usage initiation, there is need to have at least some network coverage for key-delivery, but it's enough to have very low bandwidth, even GSM control channel capacity might be enough.

\subsection{Customer terminal}
\label{customer_terminal}
\index{customer terminal}
\index{terminal, customer}

Networked wireless customer terminal; PC, tablet, phone, iot-device,.. could and should be capable to use same dependency resolution and capability installation mechanism starting from government regulated wireless communication device e.g. cellular modem created remote pre-execution-boot capability. Installation of whole software stack should be capable to do directly from the asset tree, including actual client application mentioned above has to be supported. Into cellular modem included key services are connected to user service provider, which is normally customers bank. Through bank customer get the financing and bookkeeping. Bookkeeping from purchases, purchase guarantees, certificates, certificate mortgage and protected content key generation services, so if he want to run digitally protected protected. Then with the client application you can add more abilities, or purchase something totally different than software, media and communication services.  For example those grocery purchases can be done directly from client application. Some additional storage memory might be useful for users terminal to have, at least for locally stored assets.

\subsection{Original vendor label and certificate creation service}
\label{label_and_certificate_service}
\index{label and certificate service}
\index{service, label}
\index{service, certificate}

Vendors need mobile device software used to read container, pallet, container codes, time and place (GPS) linked to electric labels on asset. Bulk delivery certificates are formed as pallet lists and container content lists which all signed by vendor to from certificate tree. Certificate tree which can be sold. If sold as exported goods we could have "tax, toll, environmental, and social payment products" in asset vendor can purchase for lots meant to be exported to form asset ready for export.

For certificate delivery to trading or banking services used care from actual trading there is need to have possibility to sign certificate on device and encrypt it with public key cryptography for transfer so that receiving end can verify that certificate is coming directly from vendor.

On the shop we should have electric labels showing product lot information. Customers need software or web page what to use to check what is put in the cases of lot, who has filled it and from where. With this electric service set we should get rid off printed labels which are pain in ass in automated delivery chain. Printers do not work and labels do no hold on plastic cases, causing problems in automation, and it is just adding unnecessary cost.

\subsection{Authority terminal services}
\label{authority_terminal_service}
\index{authority terminal service}
\index{terminal service, authority}

Based to customer terminal setup additional software tools are offered for tolls to verify delivery and do they work, e.g. check that taxes and tolls required are paid and connected to asset. Authority can then sign electric delivery having tax and toll payments attached with they own signature forming an export ready asset. Same tools can be used to check imported goods that tolls, taxes, operating costs are paid for imported lot. And authority can sign created asset ready for import.

\subsection{Value account service}
\label{value_account_service}
\index{value acount service}
\index{service, value account}

Existing banking service has to be extended with the needed tools and services. Value account most needed capability is to maintain resellable certificates and offer certificate mortgage services against key generation.
Resellable merchandise certificate mortgage and key services
To get software market to work there is need to take software media and key service under the third party, which guarantee SW availability and key generation services instead vendor. This protects product value and offers key services for the software customer. Software unencrypted originals are archived under the digital preservation legal deposit\cite{EULegalDepositScheme} terms and spirit under national archive services, having copies on each country products are published. Protected copies are stored to asset archive and banks are running key management service for customer. Customer could select bank, but he need these services from some bank to use protected products. Actual trade is done by selling vendor certificates, which are then transferred to customer's value account. When used resellable certificate is mortgaged against the key generation, and key is delivered into customer terminal cellular modem included key services and used to enable capabilities on device. And when usage is ended same service can remove key from device and release certificate from value account for resale on market. Consumables certificates are fully used on delivery. Rentable related certificates are removed from value account after rent period end, etcetera. For the the key services is enough to have SMS connection, actually meaning GSM slow control channel connection, which is used for SMS.
Trading service
Trading service can be built under banking operations and services. It will show asset, against client needs resolution is done, and product availability information in form of certificates against the purchase is done. Trading service could also be purchased service from trade operator willing to invest this business area. Independent trading service operator is most preferred because then operation can be legally guided and optimized for purpose. There could be foundation based setup doing trading service setup and technology development.
Optimization for grocery bulk
It might not be wise to take bulk groceries under trade actions, just deliver needed results, like guarantee certificates etc. under value account. One possibility is that grocery store introduce sold products to trade system and bank bookkeeping after the sale. At lest those product certificates which are resellable should be booked to trade system. This may need separate sub agreement to go through area in detail.

Automarketing is reimbursement capability for platform marketing money what each vendor has to pay for platform. Money is used for loyalty programs where customer created bundle definitions used as purchase lists so that he remembers to purchase all. We give customer hes fair share of hes semi-automated purchases as discount from price. Customer could also create  bundle definition from hes earlier purchase. While replaying purchase discount is calculated. Customer involvement to marketing is increased with possibility to get marketing money out from other users purchases by creating bundles and acting as influencer user. Same mechanism is used to pay influencer-customer from bundles he manages to market on social media or any other way. Bundle itself includes automatic statement that configuration is customer created and  vendors are giving guarantees only for single items as stated on product official documentation on offered asset tree. Customer created bundle configuration itself has no guarantee from fit for purpose. Influencer-customer gains marketing money he/she may get out as services, products or money, depending from local laws,  to support hes activity. Influencer-customers are encouraged to incorporate they efforts into ecosystem as vendor having product responsibility and they configuration fit for purpose, which means that they have official configuration describing document in asset. Then they can get pure money for they account from they marketing work what they are done. Bundle definitions are directly stored to slow archival medium, and on cloud based systems those bundles which are significantly sold are lifted to cloud edge for fast service response times. See mRACEIIQ (measure Reach Act Convert Engage Involve Incorporate Query) figure \ref{fig:mRACEIIQ} on page \pageref{fig:mRACEIIQ} forming closed feedback measurement and participant inclusion loop for ecosystem.

\begin{figure} %[H] %\usepackage{float}
 \begin{center}
  %\epsfig{figure=figures/automark.eps,width=4cm}
  \epsfig{figure=figures/automark.eps,width=\textwidth}
  \caption{Marketing money sharing for members inclusion, incorporation and viral automarketing}
  \label{fig:automarket}
 \end{center}
\end{figure}

\subsection{Automarketing algorithm}
\label{automarketing}
\index{automarketing}

Each product in asset pay marketing money and will get refund comparing to it's significance in configuration. Significance is defined as inverse of node depth in configuration. Rest marketing money goes to each configuration root node as node's own marketing money. When several configurations form new bigger configuration then algorithm counts leaf configurations first and then proceeds towards upper configurations setting and counts sub-configuration roots as upper configuration leaf, from counted marketing money inverse from node depth is returned to configuration creator, rest is going to upper configuration root.

\section{Responsibility sustainability}
\label{responsibility_sustainability}
\index{responsibility sustainability}

While this digitization improves peoples capability to utilize all possible resources and consume more, we have to have mechanisms to maintain sustainability and fair trade? Or what you say?

Big trend is the customers environmental awareness rise which is lifting head even peoples are lazy and existing  big players may try to hinder change due they own  benefit dependency to some clearly outdated way to live here in this single earth we have. Or if you have some spare earths to consume, please tell that for others too, then they may be able to continue as nothing has happened.

Consumption redirection from pure material consumption towards more social services and different kind of development activities tweaking economy towards more sustainable technology use and life style fitting to existing sustainable capabilities we have. Old way living may require heavy investment to sea, solar, wind, geothermal, nuclear energy to create fuels from air and water without fossil fuel resources utilization. And new creative ways of farming food is also needed.

Risk is that autocratic chieftains drive they own very short term benefits, peoples resisting change are supporting these chieftains, against they own and they children's benefit. Then result is global repetition from what happened on Easter Island on earth last resort where we get clan chieftain moai repetition when all key resources are used and system collapse.

\subsection{Product tracking}
\label{product_tracking}
\index{product tracking}
\index{tracking, product}

Product tracking along delivery route is one way to make sure that product fulfills some socioeconomical and environmental compatibility with the standards and certificates. Product tracking from production site by utilizing worker and/or en\-tre\-pre\-neurs own mobile device,  logistic operator signature authorities signatures, distribution center signature from receiving. Distribution center splitted assets including reference to original incoming signed delivery. Work payments inclusion into asset as already paid components of product which price is cumulated to upper level by payer, or as direct payment requiring parts of delivery in asset where all sub parts are paid same time. Geolocation marks can be used to check original harvesting area.

\subsection{Standards and certification}
\label{standards_and_certification}
\index{standards and certification}

What ever standards or certifications vendor delivered products are following, there can be they own statement from fulfillment and reference to inspection body given digitally signed certificate and/or reference to inspection body service from where to check certificate validity. It would be interesting question to check what is cost to check if customer purchasing some standard fulfilling product could be deserved right to get standard text visible in electric form from asset when purchasing product. If this cost is possible to embedded into platform cost and make all standards available by default, most peoples do not want to download everything, which means it's more available for creation and quality check reasons. Anyhow all standards are brought to platform so that they can be referred and standard texts are also available for purchase as any other product.

\subsection{Quality tracking}
\label{quality_tracking}
\index{quality tracking}
\index{tracking, quality}

All parties participating to product brand creation have to have balanced visibility to vendors products and should be able to comparing product reliability: how clear and accurate given information is to others. We have to avoid situation we can see from some current existing trade platforms, where product return rate is not told and user comments from returned product returning reasons are removed. Yep, it's somehow fair for vendor that if they return money from returned product then customer complains are not shown, but same time it allows to vendor continue to deliver shitty products, which do not fulfill even vendor own marketing specs, ahead to new unaware customers, expecting that some customers have lower requirements or they are too lazy or too busy to return product on time to get money back, even product doesn't fulfill customer needs. This is irresponsible use of everyone's resources, and has to forced to cease. Vendor has to have product/service matching to asset documentation and certificate's delivery promise.

To get products comparable with the each other we have to have measures:
\begin{itemize}
\item Product official asset documentation and certificate information fulfillment on delivery. Does product fulfill or exceed it's specs on unboxing event?
\item Promised guarantee time and promised support time for sold product/service?
\item Error tracking. Does product/service have single, occasional or repeating malfunction or operation outside of specs? Does error occur during first two weeks from purchase/unboxing, during guarantee time or during normal expected lifetime? What is error fixing rate for noted errors?
\item Product return rate and reasons for return, possibly as link to error tracking having detailed product version information.
\item Reputation in known customer responses, one +/- vote per response per customer.
\end{itemize}

From measures automatically created quality value could be used to comparison, comparing to known competing reference products quality values, and for products quality/price ratio evaluation.

\subsection{Customer protection}
\label{customer_protection}
\index{customer protection}

There is lot of people who have used to use cash money and get control to they spending from physical money transfer during purchase. With this electric possibility to purchase they may loose control to they spending and to whole life. One possibility to guarantee some continuity is to have income account where is automatic sliding year daily taxation with the negative tax values for days peoples didn't have any income. When peoples get money, perhaps afterwords, then payment is distributed over the period time payment is cumulated and taxation has been repeated. Corrected tax is taken from payment during correction and rest is paid to account. This way we guarantee that peoples have some money on every day and even all short jobs can increase total income without some governments favored stepwise rules which cause incentive traps here and there. See details Daily Tax booklet\cite{DayTax}
%attached as appendix starting from page numbered as \pageref{dailytax}
.

\subsection{Integration to housing and logistics}
\label{integration}
\index{integration, housing}
\index{integration, logistics}
Integration to housing, even it's new investment, may offer savings in long run if widely accepted an used. As block of flats may now have post box closures, those closures can be expanded to hold temperature and gas controlled closures for food delivery. This can reduce private car usage when one delivery car can bring several deliveries at once. Anyhow the whole delivery logistic: containers, trucks, pallets, cages, cases, temperature, moisture, humidity and gas  zones and automatic handling machinery requirements has to be taken in account. Now there are lot of partly optimized solutions for different usages and zones and dimensions are also differing from hand carried imperial unit size cases to automation optimized metric cases. In generally there is lot of things to do which are lacking public discussion starting from different needs and how to bring whole delivery chain to fluent, mostly automated state. We could start to think requirements from initial producer and from end user perspective and then add requirements from automation and logistics. Because end user is paying then it's maybe best point to start. Most of customers purchase most often daily products. What are they requirements, how they think  from new initiatives, for example from two door closure through detached house wall. Closure having five temperature zones from bottom to top; frozen, cold, chilly, room temperature, hot, about from \textminus25\textdegree C to +75\textdegree C, as producer and food control officials recommend for temperature controlled delivery chain. Meaning that delivery service can bring standard size automation optimized recyclable boxes through outer wall door to closure. Whole closure and boxes has to be productized for both housing and logistics industry and if possible be compatible with freezer, fridge, oven, dishwasher and standard room closure. It is then more question from what is feasible. Current infrastructure installed base is huge. In this document we have to leave this integration discussion to this level and concentrate to actual frame agreement and tradesystem. Frame agreement could then define details later on when feasible requirements are noticed. 

\subsection{Wholesale integration}
\label{wholesale}
\index{wholesale}
Wholesale is basically vendor, but now days usually integral part of retail chain. Integration willingness and timetable may wary significantly. Anyhow light integration where sold goods having guarantee or contain on market resellable items, should be listed on customer value account and trade operator sold items hash list on archive, so that if coming to resale, tradeoperator can trust item existence and authenticity. Deeper integration should be done as integrating customer and vendor roles for wholesale inbound and outbound, as well there is possibility to have invisible layers for business to business connections, see figure \ref{fig:layering} on page \pageref{fig:layering}.

\subsection{General inclusion, integration to society}
\label{inclusion}
\index{inclusion}

Most of technology hast to be standardized to and offered openly for the others too as a service to get wider acceptance on society. Trade technology provider and operator is significant market player it is good to prepare to offer created banking and trading technologies for the other players as technology services. Good to prepare before forced to do so due market position. Cooperation and support are given from integration possibilites. For end user visible application interfaces (API) etc. are strightly controlled, but unvisible B2B connections can freely utilize existing or whatever connections as long as public asset offer promises can be met by using those.

%-END OF INCLUDE FILE----------------------------------------------------------

